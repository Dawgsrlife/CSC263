\documentclass[12pt]{article}

\usepackage[utf8]{inputenc}
\usepackage{amssymb}
\usepackage{amsthm}
\usepackage{amsmath}
\usepackage{graphicx}
\usepackage{setspace}
\usepackage{geometry}
\usepackage{xcolor}
\usepackage{listings}
\usepackage{cancel}
\usepackage{comment}

\graphicspath{{Images/}}
\onehalfspacing
\geometry{letterpaper, portrait, includeheadfoot=true, hmargin=1in, vmargin=1in}

\definecolor{myblue}{RGB}{0, 128, 255}
\definecolor{mygreen}{RGB}{34, 139, 34}
\definecolor{myorange}{RGB}{255, 140, 0}
\definecolor{mygray}{RGB}{128, 128, 128}
\definecolor{mypurple}{RGB}{148, 0, 211}
\definecolor{myred}{RGB}{255, 69, 0}

\lstset{
    language=Python,
    basicstyle=\ttfamily\small,
    keywordstyle=\color{myblue}\bfseries,
    commentstyle=\color{mygreen}\itshape,
    stringstyle=\color{myorange},
    numberstyle=\color{mygray},
    identifierstyle=\color{mypurple},
    showstringspaces=false,
    breaklines=true,
    numbers=left,
    numbersep=5pt,
    frame=single,
    rulecolor=\color{mygray},
    moredelim=[is][\color{myred}]{@@}{@@},
}

\begin{document}
\renewcommand{\familydefault}{\rmdefault}

\begin{titlepage}
    \null % This is a TeX command that does nothing but is necessary for vfill to work correctly
    \vfill
    \begin{center}
        {\fontsize{35}{48}\selectfont \bfseries CSC263 Tutorial \#1 Exercises}
        \vspace{20pt} \\
        {\LARGE Analyzing the Worst, Best, and Average Max Function} \\
        \vspace{20pt}
        \textbf{Alexander He Meng}
        \vspace{8pt}
        \\ Prepared for January 10, 2025
    \end{center}
    \vfill
\end{titlepage}


\section*{CSC263H5 – Tutorial 10: Minimum Cycler and MST Uniqueness}

\subsection*{1. Minimum Cycler Algorithm}

\textbf{Problem:} Given a weighted connected undirected graph \( G = (V, E) \), find a set \( S \subseteq E \) of minimum total weight such that every cycle in \( G \) contains at least one edge in \( S \).

\textbf{Hint:} The algorithm can be described in one short sentence.

\subsubsection*{Algorithm:}

Compute a \textbf{Maximum Weight Spanning Tree (MWST)} \( T \subseteq E \), and set \( S = E \setminus T \).

\subsubsection*{Correctness:}
\begin{itemize}
    \item Every cycle in \( G \) must contain at least one edge not in any spanning tree.
    \item Removing all non-tree edges eliminates all cycles, ensuring \( S \) covers all cycles.
    \item To minimize the total weight of \( S \), maximize the weight of \( T \), since:
    \[
    \text{weight}(E) = \text{weight}(T) + \text{weight}(S)
    \]
    \item Thus, maximizing \( \text{weight}(T) \) minimizes \( \text{weight}(S) \), yielding a \textbf{Minimum Cycler}.
\end{itemize}

\subsubsection*{Runtime:}
\begin{itemize}
    \item Kruskal’s Algorithm (modified for MWST): \( O(m \log n) \)
    \item Sorting edges: \( O(m \log n) \)
    \item Union-Find: \( O(m \alpha(n)) \) where \( \alpha(n) \) is inverse Ackermann
    \item Total: \( O(m \log n) \)
\end{itemize}

\newpage

\subsection*{2. Uniqueness of MSTs}

\subsubsection*{(a) Existence of Exactly 2 Distinct MSTs}

\textbf{Answer:} Yes, such a graph exists.

\textbf{Example:}  
Graph with 5 vertices forming a cycle plus one diagonal:
\[
V = \{v_1, v_2, v_3, v_4, v_5\},\quad E = \{(v_1,v_2), (v_2,v_3), (v_3,v_4), (v_4,v_5), (v_5,v_1), (v_1,v_3)\}
\]

\textbf{Edge weights:}
\[
w(v_i,v_{i+1}) = 1\ (\text{mod }5), \quad w(v_1,v_3) = 2
\]

\textbf{Two MSTs:}
\begin{itemize}
    \item MST 1: exclude \( (v_5,v_1) \)
    \item MST 2: exclude \( (v_2,v_3) \)
\end{itemize}
Both MSTs have total weight 4. No other spanning tree can omit the higher weight edge \( (v_1,v_3) \) and maintain minimum weight.

\subsubsection*{(b) Unique Weights \( \Rightarrow \) Unique MST}

\textbf{Proof:}
Assume for contradiction \( T_1 \neq T_2 \) are two distinct MSTs.

Let \( e \in T_1 \setminus T_2 \) be the minimum weight edge among all differences. Then in \( T_2 \cup \{e\} \), a cycle exists. Let \( e' \in T_2 \setminus T_1 \) be an edge in this cycle.

Since weights are unique and \( w(e) < w(e') \), replacing \( e' \) with \( e \) in \( T_2 \) gives a lighter spanning tree – contradiction.

\textbf{Conclusion:} \( T_1 = T_2 \). MST is unique.

\subsubsection*{(c) Non-Unique Weights, Unique MST}

\textbf{Answer:} Yes, possible.

\textbf{Example:}  
Graph: 5 vertices in a tree with 4 edges of weight 1, and one additional edge of weight 1 creating a cycle.

\textbf{Reasoning:}
Although weights are not unique, only one spanning tree avoids creating a cycle and has minimal total weight. No alternative tree can be formed due to graph structure.

\textbf{Conclusion:} Duplicate weights \( \nRightarrow \) multiple MSTs. Uniqueness can be preserved structurally.

\end{document}
