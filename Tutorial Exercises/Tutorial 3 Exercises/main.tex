\documentclass[12pt]{article}

\usepackage[utf8]{inputenc}
\usepackage{newunicodechar}
\newunicodechar{ℝ}{\mathbb{R}}
\usepackage{EngReport}
\usepackage{listings}
\usepackage{cancel}
\usepackage{comment}
\usepackage{amssymb}
\usepackage{amsthm}
\usepackage{amsmath}
\usepackage{graphicx}
\usepackage{setspace}
\usepackage{geometry}
\usepackage{xcolor}  % Required for coloring in listings

\graphicspath{{Images/}}
\onehalfspacing
\geometry{letterpaper, portrait, includeheadfoot=true, hmargin=1in, vmargin=1in}

% Define custom colors
\definecolor{myblue}{RGB}{0, 128, 255}
\definecolor{mygreen}{RGB}{34, 139, 34}
\definecolor{myorange}{RGB}{255, 140, 0}
\definecolor{mygray}{RGB}{128, 128, 128}
\definecolor{mypurple}{RGB}{148, 0, 211}
\definecolor{myred}{RGB}{255, 69, 0}

% Configure listings for Python with custom styles
\lstset{
    language=Python,             % Set language to Python
    basicstyle=\ttfamily\small,  % Use a smaller monospace font
    keywordstyle=\color{myblue}\bfseries,  % Keywords in blue and bold
    commentstyle=\color{mygreen}\itshape,  % Comments in green and italic
    stringstyle=\color{myorange},          % Strings in orange
    numberstyle=\color{mygray},            % Line numbers in gray
    identifierstyle=\color{mypurple},      % Functions and variables in purple
    morekeywords={print, len, range},      % Define additional Python keywords
    showstringspaces=false,                % Do not show spaces in strings
    breaklines=true,                       % Enable line breaking
    numbers=left,                          % Add line numbers to the left
    numbersep=5pt,                         % Space between line numbers and code
    frame=single,                          % Add a box around the code
    rulecolor=\color{mygray},              % Frame color
    moredelim=[is][\color{myred}]{@@}{@@}, % Custom inline LaTeX coloring
}

\begin{document}
\renewcommand{\familydefault}{\rmdefault}

\section*{Question 1}
Suppose we want to use a Binary Search Tree to store only keys (without any additional information), and we want to allow duplicate keys. Modify the TreeInsert algorithm to handle duplicate keys. Provide the updated algorithm and explain the changes.

\vspace{1em} % Space for typing explanation
\textbf{Changes made:} \underline{Add your explanation here.}

\begin{lstlisting}[language=Python, caption={Modified TreeInsert Algorithm}]
def TreeInsert(root, key):
    if root is None:
        return Node(key)
    if key <= root.key:
        root.left = TreeInsert(root.left, key)
    else:
        root.right = TreeInsert(root.right, key)
    return root
\end{lstlisting}

\pagebreak

\section*{Question 2}
Describe the strategy to ensure duplicate keys are not always inserted on the same side. Use a boolean flag \texttt{goLeft} in each node and explain the changes.

\vspace{1em} % Space for typing explanation
\textbf{Changes made:} \underline{Add your explanation here.}

\begin{lstlisting}[language=Python, caption={Modified TreeInsert with goLeft Flag}]
def TreeInsert(root, key):
    if root is None:
        return Node(key, goLeft=True)
    if key == root.key:
        if root.goLeft:
            root.left = TreeInsert(root.left, key)
        else:
            root.right = TreeInsert(root.right, key)
        root.goLeft = not root.goLeft
    elif key < root.key:
        root.left = TreeInsert(root.left, key)
    else:
        root.right = TreeInsert(root.right, key)
    return root
\end{lstlisting}

\pagebreak

\section*{Question 3}
Describe the strategy to randomly choose the subtree for duplicate keys. Explain the changes and provide the updated algorithm.

\vspace{1em} % Space for typing explanation
\textbf{Changes made:} \underline{Add your explanation here.}

\begin{lstlisting}[language=Python, caption={TreeInsert with Randomized Insertion}]
import random

def TreeInsert(root, key):
    if root is None:
        return Node(key)
    if key == root.key:
        if random.choice([True, False]):
            root.left = TreeInsert(root.left, key)
        else:
            root.right = TreeInsert(root.right, key)
    elif key < root.key:
        root.left = TreeInsert(root.left, key)
    else:
        root.right = TreeInsert(root.right, key)
    return root
\end{lstlisting}

\pagebreak

\section*{Question 4}
Propose a better strategy for handling duplicate keys and provide a complete algorithm with analysis.

\vspace{1em} % Space for typing explanation
\textbf{Proposed Strategy:} \underline{Describe your strategy here.}

\begin{lstlisting}[language=Python, caption={Proposed TreeInsert Algorithm}]
def TreeInsert(root, key):
    # Your custom logic here
    pass
\end{lstlisting}

\end{document}
