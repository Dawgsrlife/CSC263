\documentclass[12pt]{article}

\usepackage[utf8]{inputenc}
\usepackage{newunicodechar}
\usepackage{EngReport}
\usepackage{listings}
\usepackage{cancel}
\usepackage{comment}
\usepackage{amssymb}
\usepackage{amsthm}
\usepackage{amsmath}
\usepackage{graphicx}
\usepackage{setspace}
\usepackage{geometry}
\usepackage{xcolor}  % Required for coloring in listings

\graphicspath{{Images/}}
\onehalfspacing
\geometry{letterpaper, portrait, includeheadfoot=true, hmargin=1in, vmargin=1in}

% Define custom colors
\definecolor{myblue}{RGB}{0, 128, 255}
\definecolor{mygreen}{RGB}{34, 139, 34}
\definecolor{myorange}{RGB}{255, 140, 0}
\definecolor{mygray}{RGB}{128, 128, 128}
\definecolor{mypurple}{RGB}{148, 0, 211}
\definecolor{myred}{RGB}{255, 69, 0}

% Configure listings for Python with custom styles
\lstset{
    language=Python,             % Set language to Python
    basicstyle=\ttfamily\small,  % Use a smaller monospace font
    keywordstyle=\color{myblue}\bfseries,  % Keywords in blue and bold
    commentstyle=\color{mygreen}\itshape,  % Comments in green and italic
    stringstyle=\color{myorange},          % Strings in orange
    numberstyle=\color{mygray},            % Line numbers in gray
    identifierstyle=\color{mypurple},      % Functions and variables in purple
    morekeywords={print, len, range},      % Define additional Python keywords
    showstringspaces=false,                % Do not show spaces in strings
    breaklines=true,                       % Enable line breaking
    numbers=left,                          % Add line numbers to the left
    numbersep=5pt,                         % Space between line numbers and code
    frame=single,                          % Add a box around the code
    rulecolor=\color{mygray},              % Frame color
    moredelim=[is][\color{myred}]{@@}{@@}, % Custom inline LaTeX coloring
}

\begin{document}
\renewcommand{\familydefault}{\rmdefault}

\begin{titlepage}
    \null % This is a TeX command that does nothing but is necessary for vfill to work correctly
    \vfill
    \begin{center}
        {\fontsize{35}{48}\selectfont \bfseries CSC263 Tutorial \#8 Exercises}
        \vspace{20pt} \\
        {\LARGE Bipartite Graphs and Algorithms} \\
        \vspace{20pt}
        \textbf{Alexander He Meng}
        \vspace{8pt}
        \\ Prepared for March 14th, 2025
    \end{center}
    \vfill
\end{titlepage}

\pagestyle{fancy}
\fancyhf{}
\setlength{\headheight}{30pt}
\renewcommand{\headrulewidth}{0.4pt}
\renewcommand{\footrulewidth}{0.4pt}
\lhead{\large \textbf{CSC263 UTM} \\ \textbf{Tutorial 2 Exercises}}
\rhead{\large \textbf{Winter 2025} \\ \textbf{Prepared for Jan 2025}}
\rfoot{\textbf{Page \thepage}}
\lfoot{}
\pagebreak
\normalsize

\section*{Question 1}
\textbf{(a) Checking for duplicate names in an unsorted list}

To check if any two students have the same name, we can use a hash table where:
- The keys are student names.
- The values are counts of occurrences.

\textbf{Algorithm:}
1. Initialize an empty hash table.
2. Iterate through the list:
    - If the name is already in the hash table, return \texttt{True} (duplicate found).
    - Otherwise, insert the name into the hash table.
3. If no duplicates are found, return \texttt{False}.

\textbf{Complexity:} \(O(n)\), assuming hash table operations are \(O(1)\).

\textbf{Is a hash table the best choice?} Yes, because it provides an optimal \(O(n)\) solution compared to \(O(n^2)\) for nested loops.

\textbf{(b) Sorting names using a hash table}

A hash table does not inherently support sorting. Instead, a more suitable approach is:
1. Insert names into a balanced BST (\(O(n \log n)\)).
2. Use quicksort or mergesort (\(O(n \log n)\)).

A hash table is \textbf{not} appropriate here because hashing does not maintain order.

\pagebreak
\section*{Question 2}
\textbf{(a) Open Addressing Insertions}

\textbf{Linear Probing:} \(h(k, i) = (h'(k) + i) \mod m\)

\textbf{Quadratic Probing:} \(h(k, i) = (h'(k) + i^2) \mod m\)

\textbf{Double Hashing:} \(h(k, i) = (h'(k) + i \cdot h''(k)) \mod m\)

\textbf{(b) INSERT Pseudocode}
\begin{lstlisting}
INSERT(T, k):
    i = 0
    repeat:
        j = h(k, i)
        if T[j] is empty:
            T[j] = k
            return j
        i = i + 1
        if i == m:
            return ERROR  // Table full
\end{lstlisting}

\textbf{(c) SEARCH Pseudocode}
\begin{lstlisting}
SEARCH(T, k):
    i = 0
    repeat:
        j = h(k, i)
        if T[j] == k:
            return j
        if T[j] is empty:
            return NOT FOUND
        i = i + 1
        if i == m:
            return NOT FOUND
\end{lstlisting}

\textbf{(d) DELETE Discussion}
Simply replacing the deleted element with \texttt{NIL} disrupts probing sequences. A common solution is to use a special \texttt{DELETED} marker, which allows search and insertion to function correctly. However, excessive deletions degrade performance, necessitating periodic rehashing.

\section*{Hash Table Insertion Algorithm}
The following Python implementation demonstrates an insertion function for open addressing in a hash table:

\begin{lstlisting}
hashtable = [empty] * 11

def INSERT(k):
    for i in range(0, 12):
        bucket = h(k, i)
        if hashtable[bucket] is empty:
            hashtable[bucket] = k
            return
    # Error case when the table is full
    print("Error: Hash table is full")

# Example usage
h(k, i) = 0 + i
INSERT(1)
INSERT(2)
DELETE(1)
SEARCH(2)
print(hashtable)  # Output: [DELETED, 2, NIL, NIL, NIL, NIL]
\end{lstlisting}

This code implements linear probing for collision resolution, where the function \texttt{h(k, i)} determines the next available slot in case of a collision.

\end{document}